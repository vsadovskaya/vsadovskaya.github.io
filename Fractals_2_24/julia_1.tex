
\documentclass[12pt]{article}
\usepackage{amssymb,amscd}
\usepackage{euscript}

\textwidth19cm
\hoffset=-2cm
\addtolength{\textheight}{2cm}
%\leftmargin 3cm
%\rightmargin 3cm
\voffset=-3cm

\def\R{\mathbb R}
\def\C{\mathbb C}

\begin{document}
\setlength{\parindent}{0pt}

{\Large
\hskip4cm 
{ \bf {Julia Sets}} 

\vskip.6cm

{\large
For a fixed complex number $c, \,$ let\,\, $f_{c\,} (z)= z^2+c.$ \hskip4cm
\vskip.3cm  

For each $z\in \C$ we consider its orbit, \,$\{f_c^{n}(z)\}=\{\,z, \, f_c(z), \, f_c^2(z), \,...\,\}$.
\vskip.1cm 
For some  $z\in C$, \,  $\{f_c^{n}(z)\}$ is bounded, and for some it is not. 
\vskip.3cm 
The {\bf Julia set} $\mathbf{J(f_c)}$ is the  {\bf boundary}\, of the set of points   
\vskip.1cm 
whose orbits under the map $f_c$ are bounded.  

\vskip.6cm

{\bf Example.} $\;{\bf c=0},\;\,$ $f_c(z)=z^2$.\, 
\vskip.1cm
The orbit of $z\;=\{z, z^2, z^4,  ...\}\,$ is bounded $\,\Leftrightarrow$ $\,|z|\le 1$. 
\vskip.1cm
So $J(f_0)$ is the unit circle $\{z:\; |z|=1\}$
\vskip.5cm



}
\end{document}











